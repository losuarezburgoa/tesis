\chapter{Fundamentos de estabilidad de laderas.}
Los metodos de analisis de estabilidad de laderas por equilibrio limite, que tradicionalmente se efectuan sobre un plano bidimensional, pueden llevarse a un plano tridimensional.
De esta manera, cada dobela (slice) de suelo  que anteriormente pertenecia a un perfil altitudinal, pasa a ser una columna perteneciente a una ladera. Dicha columna posee un volumen, una masa y se considera no deformada y homogenea.

Al realizar dicha presuncion, es de esperarse que, ademas de la fuerza normal en la base de cada columna, tambien extista una fuerza de cizalla en los laterales de cada una de las columnas que compone una masa de suelo. Siendo necesario calcular el campo de esfuerzos al interior de dicha masa de suelo si se desea conocer el funcionamiento de las fuerzas de friccion entre las columnas.

De esta manera, el los calculos resultntes de la interaccion entre los laterales de las columnascolumnas y la fuerza resultante del peso de la columna (en la base de la misma). Es una de las principales diferencias entre los metodos de calculo del equilibrio limite. Las tecnicas que se usan en este trabajo, conocidas como Bishop simplificado y Fellenius tradicional, no tienen en cuenta las dichas fuerzas de interaccion lateral.

tambien es importante tener en cuenta, para el calculo del Factor de seguridad (\textit{F}) ambos metodos asumen que en el momento en que se produce un deslizamiento, este ocurre simultaneamente a lo largo de una superficie de falla.

