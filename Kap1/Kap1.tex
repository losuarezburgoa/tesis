\chapter{Introducci\'{o}n}
Tradicionalmente, desde sus comienzos en la d\'ecada de los a\~nos 30 del siglo pasado, la determinaci\'on del factor de seguridad de laderas se ha concebido como un m\'etodo para estimar la probabilidad de que un talud, escarpe o ladera presente inestabilidad o pueda ceder ante la incapacidad de los materiales que la componen para sostener su peso en estado parcial o totalmente saturado.\\

En el a\~{n}o 1937 Fellenius propone el m\'etodo de dobelas para simular la probabilidad de ocurrencia de deslizamientos tipo rotacional en macizos de suelo. Para ello se selecciona un lugar que se considera representativo del macizo, en el cual se intersecta un plano imaginario ortogonal a la direcci\'on de la cara de la ladera, para obtener un perfil de elevaci\'on bajo el cual se modelan los estratos que componen el macizo de suelo y roca.\\

Aplicado correctamente, este planteamiento ha probado ser acertado al extrapolar los an\'alisis de la zona seleccionada al macizo en caso de estudio. Distintas formulaciones matem\'aticas han sido propuestas a lo largo del siglo XX con el objetivo de simular de manera precisa la interacci\'on de fuerzas que se produce entre dobelas.

Gracias a la capacidad de c\'omputo a la que se tiene acceso hoy en d\'ia, es posible evaluar tridimensionalmente una superficie con ayuda de los Sistemas de Informaci\'on Geogr\'afica (SIG) la cual es representada por medio de un \emph{Modelo de Elevaci\'on Digital} (\textit{Digital Elevation Model}, DEM por sus siglas en ingl\'es) el cual es un archivo de imagen discretizado en p\'ixeles (i.e. archivo raster), es decir, que se compone por celdas cada una de las cuales posee un valor de elevaci\'on sobre un nivel de referencia.
\\
De esta forma, el m\'etodo de dobelas pasa a ser un an\'alisis de columnas el cual no se limita a una secci\'on infinitesimalmente estrecha, sino que es posible analizar la totalidad de la zona de inter\'es.\par

Como objetivo de este estudio se plantea modelar la estabilidad de una ladera a partir de un \emph{an\'alisis tridimensional de equilibrio l\'imite} por movimientos en masa para la cuenca hidrogr\'afica de la \emph{Quebrada La Linda} en la Vereda Monte Loro en Ciudad Bol\'ivar (Antioquia) mediante el programa Scoops 3D.
Su importancia se deriva a que la zona de estudio se encuentra altamente poblada (memoria geologica) con abundancia de cultivos agr\'icolas que ha presentado ocurrencia documentada de movimientos en masa tipo rotacional.\\

Para llevar a cabo dicho estudio se plantean como objetivos espec\'ificos.

\begin{itemize}
    \item Proponer una metodolog\'ia para la obtenci\'on de DEM y par\'ametros de resistencia a usar en el software Scoops 3D.
    \item Producir un mapa de la zona de estudio sobre el cual puedan verse los factores de seguridad y su distribuci\'on en La Vereda Monteloro.
    \item Hacer el control de calidad a la informaci\'on SIG y calcular la distribuci\'on de factores de seguridad obtenidos.
    \item Interpretar la distribuci\'on del factor de seguridad obtenida y su correlaci\'on con los factores que controlan su variabilidad.
\end{itemize} 

Inicialmente se han realizado pruebas piloto para comprobar el correcto funcionamiento del software con los datos fuente obtenidos.
El resultado del uso del software Scoops3D es una im\'agen raster en tonos de gris en la cual el valor de cada p\'ixel corresponde al factor de seguridad calculado por el m\'etodo de Bishop para la totalidad de la zona trabajada. Finalmente, se podr\'a determinar la  correlaci\'on que existe entre los factores de seguridad obtenidos y las variable tenidos en cuenta, como lo son: pendiente, cohesi\'on de los materiales, resistencia al corte directo y humedad.
\\

