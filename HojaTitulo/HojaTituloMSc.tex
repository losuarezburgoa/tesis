%\newpage
%\setcounter{page}{1}
\begin{center}
\begin{figure}
\centering%
\epsfig{file=HojaTitulo/EscudoUN.eps,scale=1}%
\end{figure}
\thispagestyle{empty} \vspace*{1.5cm} \textbf{\huge
An\'{a}lisis tridimensional de equilibrio l\'{i}mite por movimientos en masa para la cuenca hidrogr\'{a}fica de la quebrada La Linda en la vereda Monte Loro en Ciudad Bolivar (Antioquia) mediante el programa Scoops 3D}\\[4.0cm]
\Large\textbf{Juan Felipe Luj\'{a}n Rivas}\\[4.0cm]
\small Universidad Nacional de Colombia\\
Facultad de Minas, Departamento de ingenier\'{i}a Civil\\
Medell\'{i}n, Colombia\\
2019\\
\end{center}

\newpage{\pagestyle{empty}\cleardoublepage}

\newpage
\begin{center}
\thispagestyle{empty} \vspace*{0cm} \textbf{\huge
An\'{a}lisis tridimensional de equilibrio l\'{i}mite por movimientos en masa para la cuenca hidrogr\'{a}fica de la quebrada La Linda, en la vereda Monte Loro en Ciudad Bolivar (Antioquia), mediante el programa Scoops 3D}\\[2.0cm]
\Large\textbf{Juan Felipe Luj\'{a}n Rivas}\\[2.0cm]
\small Tesis o trabajo de grado presentada(o) como requisito parcial para optar al
t\'{\i}tulo de:\\
\textbf{ Magister en Ingenier\'{\i}a Geotecnia}\\[1.5cm]
Director(a):\\
Ph.D. Ludger O. Suarez-Burgoa\\[1.0cm]
L\'{\i}nea de Investigaci\'{o}n:\\
Estabilidad de Laderas\\
Grupo de Investigaci\'{o}n:\\
Grupo de Investigaci\'{o}n en Geotecnia (Medell\'in)\\[0.5cm]
Universidad Nacional de Colombia\\
Facultad de Minas, Departamento de Ingenier\'{i}a Civil\\
Medell\'in, Colombia\\
2019\\
\end{center}



\newpage
\thispagestyle{empty} \textbf{}\normalsize
\\\\\\%
\addcontentsline{toc}{chapter}{\numberline{}Agradecimientos}\\\\
\begin{figure}[H]
\centering
\includegraphics[trim={0 0.1cm 2.2cm 0},clip,scale=0.8]{img/dedicatoria/dedicatoria.pdf}
\end{figure}

\newpage{\pagestyle{empty}\cleardoublepage}

\newpage
\textbf{\LARGE Resumen}\\
En esta investigaci\'on se demuestra la aplicaci\'on del m\'etodo de an\'alisis de equilibrio l\'imite Bishop simplificado, pero aplicado en el espacio tridimensional. Esto aplicado en una zona conocida popularmente como Vereda Monteloro, en el municipio de Ciudad Bolivar, departamento de Antioquia. Para dicho procedimiento se utilizan como insumos: informaci\'on geogr\'afica y de elevaci\'on contenida en un modelo de elevaci\'on digital (DEM por sus siglas en ingl\'es); par\'ametros de resistencia, espec\'ificamente: el \'angulo de fricci\'on y la cohesion; todos ellos obtenidos de ensayos de laboratorio ejecutados sobre muestras recolectadas en la zona de estudio. Como resultado se obtiene un mapa de calor de la zona trabajada en el cual se logra apreciar la distribuci\'on de factores de seguridad (que descienden hasta \(1.46\)) y su relaci\'on con factores como la pendiente y los mismos par\'ametros de resistencia. Adicionalmente se obtiene informaci\'on quantitativa, como el total de superficies de falla evaluados \(1\,665\,954\), as\'i como el vol\'umen y la masa estimados del cuerpo de suelo con menor factor de seguridad encontrado, siendo estos valores \(211.180\text{m}^{3}\) y \(534.24\,\text{kg}\) respectivamente.
Se detalla y ejemplifica el uso de \emph{Scoops3D} como herramienta computacional para la generaci\'on del mapa de distribuciones de factores de seguridad. 
\addcontentsline{toc}{chapter}{\numberline{}Resumen}\\\\
