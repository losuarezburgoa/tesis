%\newpage
%\setcounter{page}{1}
\begin{center}
\begin{figure}
\centering%
\epsfig{file=HojaTitulo/EscudoUN.eps,scale=1}%
\end{figure}
\thispagestyle{empty} \vspace*{2.0cm} \textbf{\huge
An\'{a}lisis tridimensional de equilibrio l\'{i}mite por movimientos en masa para la cuenca hidrogr\'{a}fica de la quebrada La Linda en la vereda Monte Loro en Ciudad Bolivar (Antioquia) mediante el programa Scoops 3D}\\[6.0cm]
\Large\textbf{Juan Felipe Luj\'{a}n Rivas}\\[6.0cm]
\small Universidad Nacional de Colombia\\
Facultad de Minas, Departamento de ingenier\'{i}a Civil )\\
Medell\'{i}n, Colombia\\
2017\\
\end{center}

\newpage{\pagestyle{empty}\cleardoublepage}

\newpage
\begin{center}
\thispagestyle{empty} \vspace*{0cm} \textbf{\huge
An\'{a}lisis tridimensional de equilibrio l\'{i}mite por movimientos en masa para la cuenca hidrogr\'{a}fica de la quebrada La Linda en la vereda Monte Loro en Ciudad Bolivar (Antioquia) mediante el programa Scoops 3D}\\[3.0cm]
\Large\textbf{Juan Felipe Luj\'{a}n Rivas}\\[3.0cm]
\small Tesis o trabajo de grado presentada(o) como requisito parcial para optar al
t\'{\i}tulo de:\\
\textbf{ Magister en Ingenier\'{\i}a Geotecnia}\\[2.5cm]
Director(a):\\
Ph.D. Ludger O. Suarez. Burgoa\\[2.0cm]
L\'{\i}nea de Investigaci\'{o}n:\\
Estabilidad de Laderas\\
Grupo de Investigaci\'{o}n:\\
Grupo de Investigaci\'{o}n BIMs (Blocks in Matrix)\\[2.5cm]
Universidad Nacional de Colombia\\
Facultad de Minas, Departamento de Ingenier\'{i}a Civil\\
Medell\'in, Colombia\\
2017\\
\end{center}

\newpage{\pagestyle{empty}\cleardoublepage}

\newpage
\thispagestyle{empty} \textbf{}\normalsize
\\\\\\%
\textbf{(Dedicatoria o un lema)}\\[4.0cm]

\begin{flushright}
\begin{minipage}{8cm}
    \noindent
        \small
        Su uso es opcional y cada autor podr\'{a} determinar la distribuci\'{o}n del texto en la p\'{a}gina, se sugiere esta presentaci\'{o}n. En ella el autor dedica su trabajo en forma especial a personas y/o entidades.\\[1.0cm]\\
        Por ejemplo:\\[1.0cm]
        A mis padres\\[1.0cm]\\
        o\\[1.0cm]
        La preocupaci\'{o}n por el hombre y su destino siempre debe ser el
        inter\'{e}s primordial de todo esfuerzo t\'{e}cnico. Nunca olvides esto
        entre tus diagramas y ecuaciones.\\\\
        Albert Einstein\\
\end{minipage}
\end{flushright}

\newpage{\pagestyle{empty}\cleardoublepage}

\newpage
\thispagestyle{empty} \textbf{}\normalsize
\\\\\\%
\addcontentsline{toc}{chapter}{\numberline{}Agradecimientos}\\\\
\begin{figure}[H]
\centering
\includegraphics[trim={0 0.1cm 2.2cm 0},clip,scale=0.8]{img/dedicatoria/dedicatoria.pdf}
\end{figure}

\newpage{\pagestyle{empty}\cleardoublepage}

\newpage
\textbf{\LARGE Resumen}
\addcontentsline{toc}{chapter}{\numberline{}Resumen}\\\\
